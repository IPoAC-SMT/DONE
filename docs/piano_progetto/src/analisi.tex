\section{Analisi}
\subsection{Componenti}
Di seguito si analizzeranno in maniera più dettagliata le singole componenti, descrivendo come si intende realizzarle. 
\paragraph*{Tecnologie utilizzate} Complessivamente, il progetto sarà realizzato utilizzando le seguenti tecnologie: 
\begin{itemize}
    \item Linguaggio \textbf{C}, con l'ausilio della libreria \href{https://github.com/raysan5/raylib}{\texttt{raylib}} per quanto concerne l'interfaccia grafica;
    \item \textbf{Docker}, per la virtualizzazione di nodi e router;
    \item \textbf{OpenVSwitch} per la virtualizzazione di switch;
    \item Linguaggio \textbf{Python 3} per la realizzazione della CLI.
\end{itemize}
L'assenza di ulteriori dipendenze\footnote{Diamo per scontato che un dispositivo Unix-based sia dotato di editor di testo e terminale.} assicura un'\textbf{elevata portabilità} dell'applicazione, basando tutte le configurazioni di rete necessarie su strumenti già presenti in un ambiente Linux, come ad esempio i namespaces e la segregazione dello stack di rete tramite questi ultimi. 
\newline\newline
Come introdotto nella sezione precedente, il progetto è strutturato in \textbf{quattro layer distinti}:
\paragraph*{Networking} La componente di networking è interamente gestita tramite \texttt{openvswitch-switch}, \texttt{Docker} e l'interazione con i namespace dei container. La \textbf{realizzazione concreta della topologia} prevede la generazione dei nodi/switch richiesti, linkati grazie ad interfacce virtuali collegate da cavi virtuali, lavorando direttamente sul namespace del container. 
\paragraph*{Logical controller} Rappresenta il fulcro della logica di controllo. Riceve informazioni sulla topologia creata (dunque nome, tipo e collegamenti di ciascun apparato) dal livello di GUI, le formatta, parsa ed invia al livello di Networking all'\textbf{avvio della simulazione}. Gestisce inoltre il salvataggio dei progetti (topologia, collegamenti ed eventuali configurazioni dei nodi) e l'apertura di quelli salvati in precedenza. 
\paragraph*{GUI} Realizza la componente grafica con cui l'utente può interagire tramite l'applicazione, e \textbf{raccoglie i dati} generati dall'utente. Questi ultimi verranno adeguatamente incapsulati e trasmessi al Logical Controller. Si occupa inoltre di gestire i salvataggi della disposizione a livello grafico di nodi e link, appoggiandosi alle funzionalità fornite dal Logical Controller.
\paragraph*{CLI} 
Non prevista come configurazione di default, viene lanciata tramite una flag da linea di comando e si propone come \textbf{alternativa} all'approccio grafico fornito dall'interazione tra GUI e Logical Controller. Offre un \textbf{minimalismo} che, tuttavia, esclude la possibilità di usufruire di funzionalità nativamente implementate nell'approccio standard quali il salvataggio dei progetti o la traduzione automatica della configurazione visualizzata in sistema \textit{running}. 
Il principale vantaggio è quello del \textbf{testing \textit{on-the-go}} dei sistemi, permesso anche dalla versatilità e semplicità del linguaggio Python.

\subsection{Suddivisione del lavoro}
Di seguito come intendiamo suddividere il lavoro:
\begin{itemize}
    \item L'\textbf{interfaccia grafica} e la \textbf{raccolta dei dati} forniti dall'utente (dove e quali nodi collocare, e come questi vengono collegati tra di loro) viene gestita da Samuele Manclossi tramite la realizzazione di una GUI che si appoggi a strutture dati e funzioni ad hoc. Un altro compito della GUI è quello di salvare e caricare la disposizione grafica dei nodi per utilizzi futuri. Inoltre, Samuele provvederà anche a fornire una \textbf{CLI} (wrapper Python) per interagire con le funzioni di \texttt{netlib}.
    \item La riformulazione e \textbf{trasmissione} di tali dati alla libreria di rete, la gestione dello \textbf{start/stop} della simulazione (anche in caso di uscita forzata, in modo da evitare di lasciare sulla macchina ospitante container \textsl{pendenti}), oltre che il \textbf{salvataggio e parsing} delle configurazioni passate su file (così da poterle agevolmente caricare in futuro) è compito del Logical Controller implementato da Melissa Moioli.
    \item Infine, la realizzazione della \textbf{libreria di rete} (\texttt{netlib}) a cui vengono trasmessi i dati dai livelli superiori, che comunica direttamente con Docker ed OpenVSwitch consentendo la rappresentazione tramite container di ogni dispositivo, nonché il collegamento diretto a ciascuno di essi tramite \textbf{shell}, è compito di Tiziano Radicchi. Il tutto viene permesso da un accurato studio circa le possibilità di collegamento di container tramite la configurazione dei rispettivi namespace.
\end{itemize}
Nonostante la seguente suddivisione in macroaree, le scelte architetturali ed implementative sono state prese all'unanimità ed il codice prodotto è frutto della \textbf{collaborazione di tutti i membri}, talvolta portati a contribuire anche ad aree altre rispetto alla propria.

\newpage